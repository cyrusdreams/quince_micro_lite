%% LyX 2.4.3 created this file.  For more info, see https://www.lyx.org/.
%% Do not edit unless you really know what you are doing.
\documentclass[a4paper,oneside,british]{book}
\usepackage[T1]{fontenc}
\usepackage{textcomp}
\usepackage[utf8]{inputenc}
\setcounter{secnumdepth}{3}
\setcounter{tocdepth}{3}
\usepackage{babel}
\usepackage{graphicx}
\usepackage[]
 {hyperref}

\makeatletter

%%%%%%%%%%%%%%%%%%%%%%%%%%%%%% LyX specific LaTeX commands.
\pdfpageheight\paperheight
\pdfpagewidth\paperwidth


\makeatother

\usepackage{listings}
\renewcommand{\lstlistingname}{\protect\inputencoding{latin9}Listing}

\begin{document}
\title{Quince micro lite (v0.9)}
\author{CyrusDreams}
\date{December 2025}
\maketitle

\chapter*{Introduction}

The 10-cent-microcontroller, it was everywhere in the news. With a
32bit RISC-V core and useful peripherals, the CH32V003 from WCH is
suited to many projects. But such a small microcontroller needs to
be programmed via a dedicated adapter. In a forum at mikrocontroller.net,
user F.P. analyzed the WCH LinkE debugging adapter and came to the
conclusion that the binary code from WCH can be run on the USB enabled
CH32V305 microcontroller. And because the CH32V305 da do so much more
than its smaller sibling, the idea of this development board was born. 

This description is valid for the `lite' variant of said development
board an its first produced version `0.9'.

\chapter{Features of the Quince~micro~lite}
\begin{itemize}
\item board dimensions of 21~mm {*} 51~mm are the same as a \emph{RaspberryPi
Pico}, but it lacks the mounting holes, due to the use of an USB-C
connector
\item the pinout of the board is designed to match the peripherals on the
\emph{ClockworkPi Pico Calc} mainboard
\item the boards has mounting pads for a microSD card slot and a connector
for an OLED screen, if those are mounted, the board has the same features
as a \emph{Longan Nano}
\item the \emph{LinkE} debug connector is found on the PCB edge, the ``lite''
version omits the switchable target suply voltages
\item on the bottom side of the PCB, a CAN bus transceiver can be mounted,
this makes the Quince micro a small CAN bus node
\item a RGB LED can be controlled from the microcontroller, two of the LEDs
share the port with the Rx/Tx lines of the debug interface
\end{itemize}

\section{Pinout of the Quince micro}

\includegraphics[angle=90,scale=0.5]{docmentation/quince_micro_pinout}

The coloured signals match the ones on a \emph{Raspberry Pi Pico}. 

Remark: ports PB8 and PB9 will be exchanged in a later revision of
the \emph{Quince micro} to enable an additional I\texttwosuperior C
interface, compatible to the \emph{Raspberry Pi} pinout.

\section{Quince micro compared to the Longan~nano}

\includegraphics[scale=0.5]{docmentation/quince_micro_longan_nano}

On the bottom side of the \emph{Quince micro}, additional parts can
be assembled to have a similar functionality as a \emph{Longan nano}
development board:
\begin{itemize}
\item microSD slot, the port pins are also available on pins 21, 22, 24
and 25 of the \emph{Quince micro} 
\item eight-pin connector for OLED displays
\end{itemize}

\section{Quince micro as a CAN bus node}

On the bottom side of the PCB, a CAN bus transceiver can be mounted
(position U50) with its decoupling caacitor (C51, 100nF) and a four
pole SMD pin header. With the available signals CAN-high, CAN-low,
5V-supply and Ground, the Quince micro can be supplied with power
and act as a CAN bus node.

\includegraphics[scale=0.5]{docmentation/quince-micro_CAN}

Recommended are CAN-SIC transceivers (Signal Improvement Capability),
if the additional cost of approximately 0.15€ is bearable for the
intended project. These newer transceivers enable higher data rates
and/or longer wiring distances than older CAN-FD transceivers. For
\emph{Quince micro}, CAN transceivers with `standby' input and separate
IO voltage are suitable. Examples of compatible transceivers are TJA1462
, TCAN1462, TLE9371VSJ and others.

\chapter{Quince micro is a Debugger}

\section{Features of the Quince micro as a WCH LinkE Debugger}

On the short side of the PCB, opposite of the USB-S connector, there
are solder pads suitable for two sets of five pin headers. The pinout
is close to the \emph{WCH LinkE R0 1v3}; the switchable supply pins
are missing. Also not available is the ``ModeSelect'' button. Its
functionality can be controlled via software, as stated in the ``WCH-LinkUserManual''
(version V2.4).

\section{Setup of the Quince micro as a LinkE Debugger}

In the forum of mikrocontroller.net, user \href{https://www.mikrocontroller.net/user/show/fail}{Benutzer F.P. (fail)}
posted the following \href{https://www.mikrocontroller.net/topic/578672\#7974902}{Instructions}
that are similar for the \emph{Quince micro}:
\begin{enumerate}
\item Install the software from WCH installieren, commands for Linux:
\begin{enumerate}
\item \bgroup\inputencoding{latin9}
\begin{lstlisting}
git clone https://github.com/ch32-rs/wchisp
\end{lstlisting}
\leavevmode\egroup
\item \bgroup\inputencoding{latin9}
\begin{lstlisting}
git clone https://github.com/ch32-rs/wlink
\end{lstlisting}
\leavevmode\egroup
\end{enumerate}
\item Build the software (see respective dokumentation)
\item Put the microcontroller into the USB programming mode
\begin{enumerate}
\item unpower the Quince micro
\item push and hold BOOT0 switch
\item connect USB to a PC
\item release BOOT0 switch
\end{enumerate}
\item Flash the firmware from MounRiverStudio (see \href{https://www.mikrocontroller.net/attachment/672558/WCH-LinkE-APP-IAP.bin}{mikrocontroller.net)}
\begin{enumerate}
\item \bgroup\inputencoding{latin9}
\begin{lstlisting}
wchisp flash WCH-LinkE-APP-IAP.bin
\end{lstlisting}
\leavevmode\egroup
\end{enumerate}
\item restart the Quince micro: unplug USB, reconnect USB
\end{enumerate}
Now the Quince micro is ready to function as a WCH-LinkE apter. After
connecting it to another CH32 target (for example another Quince micro),
the second microcontroller can be programmed with, for example:
\begin{enumerate}
\item \bgroup\inputencoding{latin9}
\begin{lstlisting}
wlink flash blink.bin
\end{lstlisting}
\leavevmode\egroup
\end{enumerate}
The connection to a second Quince micro is done by connecting the
pins SWDIO and SWCLK of both boards.

\chapter{Developing Software for the Quince~micro}

\section{MounRiver Studio}

MounRiver Studio is the IDE preferred by WCH.

\section{Arduino}

WCH has released their own Arduino core. With it, the CH32V305-Microcontroller
of the Qince micro can be programmed with the Arduino IDE. The board
itself is not available in the IDE, but the peripherals can be addressed
as on the development board `CH32V305RBT6'. The following pins (D{*}
and A{*}) are available:

\includegraphics[scale=0.5]{docmentation/quince_micro_arduino}

\section{PlatformIO}

There is a GitHub repositry for \href{https://github.com/Community-PIO-CH32V/platform-ch32v}{CH32V}
in the PlatformIO IDE.

\section{cnlohr / ch32fun}

Developer and YouTuber CNLohr (Charles) released his own development
environment for CH32V with the name \href{https://github.com/cnlohr/ch32fun}{ch32fun}
on GitHub.

\section{Python}

There is an \href{https://github.com/r4d10n/micropython-wch-ch32v307/tree/master/ports/wch}{experimental port}
of MicroPython for the CH32V307, this should also be useable for the
CH32V305.

\chapter*{Annex}

\section*{Copyright}

This text is licensed by CC-BY-SA-4.0. Product names and other trademarks
are in the possession of their respective owners. This text makes
no claim regarding those names but needs to use them for documentation
and clarification purposes. 

The Quince micro lite board itself is licensed via CERN-OHL-S-2.0,
modifications and improvements are welcome.
\end{document}
